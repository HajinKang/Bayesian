\documentclass[9pt]{beamer}  % \documentclass[9pt]{beamer}

\usetheme{CambridgeUS} %  CambridgeUS, Madrid, Boadilla, default, Warsaw, Bergen, Frankfurt
\usecolortheme{dolphin} %  albatross beaver beetle crane default dolphin dove fly lily orchid rose seagull seahorse whale wolverine 
  
\useoutertheme[subsection=false]{miniframes}  % Used from candidacy exam
%footline=empty, footline=authorinstitute, footline=authortitle, footline=institutetitle, footline=authorinstitutetitle

% Table of contents numbering style
\setbeamerfont{section number projected}{family=\rmfamily,series=\bfseries,size=\normalsize}
\setbeamercolor{section number projected}{bg=white,fg=black}

%%%%%%%%%%%%%%%%%%%%%%%%%%%%%%%%%%%%%%%%%%%%%%%%%%%
%%%% Used formats
%%%%
%%%% JSM 2014, Boston : CambridgeUS 	beaver 9pt,  15 pages
%\setbeamertemplate{footline}[authorinstitutetitle]
% \setbeamertemplate{footline}[page number] %  if you want page numbers and using Warsaw theme%

 \setbeamertemplate{caption}[numbered]
%\setbeamercovered{transparent}
% \setbeamercovered{invisible}
\setbeamertemplate{navigation symbols}{}   % To remove the navigation symbols from the bottom of slides%

%% page number, author and institute in footline
\defbeamertemplate{footline}{authorinstitute and page number}{%
\usebeamercolor[fg]{page number in head/foot}%
\usebeamerfont{page number in head/foot}%
\hspace{1em}\insertshortauthor \hspace{0.1em} (\insertshortinstitute)\hfill%
\insertpagenumber\,/\,\insertpresentationendpage\kern1em\vskip2pt%
}
\setbeamertemplate{footline}[authorinstitute and page number]{}

%%%%  Package

\usepackage{amsthm,amsfonts,amssymb,amsbsy,amsmath}
\usepackage{graphicx, graphics}
% \usepackage{bm}         % For typesetting bold math (not \mathbold)
\usepackage{color}
%\usepackage{floatflt}
\usepackage{wrapfig} 
\usepackage{enumerate}
\usepackage{multimedia}
%\usepackage{animate}
\usepackage{multirow}
\usepackage{kotex}
\usepackage{subfigure}
%%%%  Logo

\logo{\includegraphics[height=1cm]{UploadFiles/cau_logo.png}}




%%%%%%%%%%%%%%%%%%%%%%%%%%%%%%%%%%%%%%%%%%%%%%%%%%%%%%%%%%%%%%%%%%%%%%%

% Define colors
   \definecolor{Majorelleblue}{rgb}{0.38,0.31,0.86} % Definition of color
   \definecolor{Beaverblue}{rgb}{0.2,0.2,0.7}
    
	% Inner block 
			\setbeamercolor{block title}{fg=Majorelleblue,bg=Majorelleblue}
			\setbeamercolor{block title alerted}{use=alerted text,fg=red, bg=alerted text.fg!75!bg}
			\setbeamercolor{block title example}{use=example text,fg=red, bg=example text.fg!75!bg}
%			\setbeamercolor{block body}{parent=normal, textfg=red,use=block title,bg=block title.bg!25!bg}
			\setbeamercolor{block body alerted}
							{parent=normal text,fg=red,use=block title alerted,bg=block title alerted.bg!25!bg}
			\setbeamercolor{block body example}
							{parent=normal text,fg=red,use=block title example,bg=block title example.bg!25!bg}
	%		\setbeamercolor{button border}{use=button,fg=yellow,bg=red,textfg=blue,,textbg=green}

\addtobeamertemplate{block begin}{%
  \setlength{\textwidth}{0.9\textwidth}%
}{}

\addtobeamertemplate{block alerted begin}{%
  \setlength{\textwidth}{0.9\textwidth}%
}{}

\addtobeamertemplate{block example begin}{%
  \setlength{\textwidth}{0.9\textwidth}%
}{}

 %%%%%%%%%%%%%%%%%%%%%%%%%%%%%%%%%%%%%%%%%%%%

%New command
	\newcommand{\bm}[1]{\mbox{\boldmath{$#1$}}}
	\newcommand{\bms}{\mbox{\boldmath{$s$}}}

	\newcommand{\bluedot}{\textcolor{Beaverblue}{$\cdot \ $}}
	\newcommand{\bluetri}{\textcolor{Beaverblue}{$\blacktriangleright \ $}}
	\newcommand{\bluedash}{\textcolor{Beaverblue}{-}}
	\newcommand{\bluecircle}{\textcolor{Beaverblue}{$\circ \ $}}	
	\renewcommand{\footnotesize}{\scriptsize}
%	\renewcommand{\arraystretch}{2}
%	\newcommand{\reddot}{\textcolor{Majorelleblue}{$\cdot \ $}}
%	\newcommand{\redtri}{\textcolor{Majorelleblue}{$\blacktriangleright \ $}}
%	\newcommand{\reddash}{\textcolor{Majorelleblue}{- \ }}	
%	\newcommand{\redcircle}{\textcolor{Majorelleblue}{$\circ \ $}}	

%%%%%%%%%%%%%%%%%%%%%%%%%%%%%%%%%%%%%%%%%%%%

\title[]{\huge Bayesian measures of model complexity and fit\\  (Section 7 : A model comparison criterion)}

%\subtitle{\textcolor{black}{Candidacy Examination}}

\author[Bayesian Seminar \#5]{}

\institute[HJKang, Section 7]{\Large by David J.Spiegelhalter/Nicola G.Best/Bradley P.Carlin\\ and Angelika van der Linde}

\date{February 19th (Wed), 2020} % \today will show current date.  % Alternatively, you can specify a date.

% Delete this, if you do not want the table of contents to pop up at
% the beginning of each subsection:
% \AtBeginSubsection[]
% {
%  \begin{frame}{Outline}
%    \tableofcontents[currentsection,currentsubsection]
%  \end{frame}
% }


%%%%%%%%%%%%%%%%%%%%%%%
\begin{document}

%%%%%%%%%
\begin{frame}
\titlepage
\end{frame}

%%%%%%%%%
%\begin{frame}
%   \frametitle{Outline}
%   \tableofcontents % [currentsection,currentsubsection]
%\end{frame}

%%%%%%%%%%%%%%%%%%%%%%%
%%%%%%%%%%%%%%%%%%%%%%%
%%%%%%%%%%%%%%%%%%%%%%%
\section[Intro]{Intro}
\subsection*{Intro}



%%%%%%%%%%%%%%%%%%%%%%

\begin{frame}
\frametitle{Outline}

\begin{itemize}
   \item[\bluetri] Contents of this paper (Section 1 $\sim$ 9)
   \vspace{0.2cm}
   \item[\bluetri] Section 7. A model comparison criterion
    \item[] \hspace{1.23cm} 7.1 Model selection
   \item[] \hspace{1.23cm} 7.2 Classical criteria for model comparison
    \item[] \hspace{1.23cm} 7.3 Bayesian criteria for model comparison
\end{itemize}
  \vspace{1cm}

\end{frame}



%%%%%%%%%%%%%%%%%%%%%%%
%%%%%%%%%%%%%%%%%%%%%%%
%%%%%%%%%%%%%%%%%%%%%%%
\section[Model selection]{Model selection}
\subsection*{Model selection}



%%%%%%%%%%%%%%%%%%%%%%

\begin{frame}[t]
\frametitle{Model comparison: the problem}

\begin{itemize}
   \item[\bluetri] Concept
    \item[]\bluecircle $Y_{rep}$ = independent replicate data set
    \item[]\bluecircle $\mathcal{L}(Y,\tilde{\theta})$ = loss in assigning to data Y a probability $p(Y|\tilde{\theta})$
    \item[]\bluecircle $\mathcal{L}(y,\tilde{\theta}(y))$ = 'apparent' loss repredicting the observed $y$
   \vspace{0.2cm}
   \item[\bluetri] $E_{Y_{rep}|\theta^t}[\mathcal{L}\{y,\tilde{\theta}(y)\}] = \mathcal{L}\{y,\tilde{\theta}(y)\}+c_{\Theta}\{y,\theta^t,\tilde{\theta}(y)\}$
    \item[] \hspace{0.1cm} where $c_{\Theta}$ is the 'optimism' associated with the estimator $\tilde{\theta}(y)$ (Efron, 1986)
    \vspace{0.2cm}
   \item[\bluetri] Assuming $\mathcal{L}(Y,\tilde{\theta}) = -2\log\{p(Y|\tilde{\theta})\}$, to estimate $c_{\Theta}$:
    \vspace{0.2cm}
    \begin{enumerate}
        \item \underline{Classical approach}: attempts to estimate the sampling expectation of $c_{\Theta}$
        \vspace{0.2cm}
        \item \underline{Bayesian approach}: direct calculation of the posterior expectation of $c_{\Theta}$
    \end{enumerate}    
\end{itemize}
  \vspace{1cm}

\end{frame}


%%%%%%%%%%%%%%%%%%%%%%%
%%%%%%%%%%%%%%%%%%%%%%%
%%%%%%%%%%%%%%%%%%%%%%%
\section[Classical criteria for model comparison]{Classical criteria for model comparison}
\subsection*{Classical criteria for model comparison}

%%%%%%%%%%%%%%%%%%%%%

\begin{frame}
\frametitle{Classical criteria for model comparison}


\begin{itemize}
    \item[\bluetri] Expected optimism: $\pi(\theta^t) = E_{Y|\theta^t}[c_{\Theta}\{Y,\theta^t,\tilde{\theta}(Y)\}]$
    \vspace{0.2cm} 
    \item[\bluetri] All criteria for models comparison based on minimizing $$\hat{E}_{Y_{rep}|\theta^t}[\mathcal{L}\{Y_{rep},\tilde{\theta}(y)\}] = \mathcal{L}\{y, \tilde{\theta}(y)\} + \hat{\pi}(\theta^t)$$
    \vspace{0.2cm} 
    \item[\bluetri] Efron (1986) $\pi(\theta^t)$ for the log-loss function: $\pi_E(\theta^t) \approx 2p$
    \vspace{0.2cm} 
    \item[\bluetri] Considered as corresponding to a plug-in estimate of fit + twice the effective number of parameters in the model
\end{itemize}


\end{frame}


%%%%%%%%%%%%%%%%%%%%%%%
%%%%%%%%%%%%%%%%%%%%%%%
%%%%%%%%%%%%%%%%%%%%%%%
\section[Bayesian criteria for model comparison]{Bayesian criteria for model comparison}
\subsection*{Bayesian criteria for model comparison}


%%%%%%%%%%%%%%%%%%%%%%
\begin{frame}
\frametitle{Bayesian criteria for model comparison}

\begin{itemize}
    \item[\bluetri] \textcolor{red}{\bf{AIME}}: identify models that best explain the observed data \\ {\bf but} \\ with the expectation that they minimize uncertainty about observations generated in the same way
\end{itemize}


\end{frame}



%%%%%%%%%%%%%%%%%%%%%%
\begin{frame}
\frametitle{Deviance information criterion (DIC)}


\begin{block}{\textcolor{white}{Definition}}
    \begin{eqnarray}
        DIC & = & D(\bar{\theta}) + 2p_D \nonumber\\
            & = & \bar{D} + p_D \nonumber
    \end{eqnarray}
\end{block}


\begin{itemize}
    \item[\bluetri] Classical estimate of fit + twice the effective number of parameters
    \vspace{0.2cm}
    \item[\bluetri] Also a Bayesian measure of fit, penalized by complexity $p_D$
\end{itemize}


\end{frame}


%%%%%%%%%%%%%%%%%%%%%%
\begin{frame}
\frametitle{DIC and AIC}

\begin{itemize}
    \item[\bluetri] Akaike information criterion \\ $\Longrightarrow AIC = 2p - 2\log\{p(y|\hat{\theta})\}$ \\ $\hat{\theta} = MLE$
    \vspace{0.2cm}
    \item[\bluetri] From result Section 3.2: $p_D \approx p$ in models with negligible prior information \\ $\Longrightarrow DIC \approx 2p + D(\bar{\theta})$
\end{itemize}


\end{frame}


%%%%%%%%%%%%%%%%%%%%%%
\begin{frame}[t]
\frametitle{An approximate decision theory for DIC}

\begin{itemize}
    \item[\bluetri] $c_{\Theta}\{y, \theta^t, \tilde{\theta}(y)\} = E_{Y_{rep}|\theta^t}\{D_{rep}(\tilde{\theta})\} - D(\tilde{\theta})$ \\ where $-2\log[p\{Y_{rep}|\tilde{\theta}(y)\}]$ is denoted $ D_{rep}(\tilde{\theta})$
    \vspace{0.2cm}
    \item[\bluetri] $c_{\Theta} = E_{Y_{rep}|\theta^t}\{D_{rep}(\tilde{\theta})-D_{rep}(\theta^t)\} + E_{Y_{rep}|\theta^t}\{D_{rep}(\theta^t)-D(\theta^t)\} + \{D(\theta^t)-D(\tilde{\theta})\}$
    \vspace{0.2cm}
    \item[\bluetri] $\mathcal{L}_1(\theta,\tilde{\theta}) \approx E_{Y_{rep}|\theta}\{-2(\tilde{\theta}-\theta)^TL'_{rep,\theta}-(\tilde{\theta}-\theta)^TL''_{rep,\theta}(\tilde{\theta}-\theta)\}$ \\ where $L_{rep,\theta} = \log\{p(Y_{rep}|\theta)\}$
    \vspace{0.2cm}
    \item[\bluetri] Since $E_{Y_{rep}|\theta}(L'_{rep,\theta}) = 0$ \\
    $\mathcal{L}_1(\theta,\tilde{\theta}) \approx tr\{I_{\theta}(\tilde{\theta}-\theta)(\tilde{\theta}-\theta)^T\}$ \\ where $I_{\theta} = E_{Y_{rep}|\theta}(-L''_{rep,\theta})$ is the assumed Fisher information in $Y_{rep}$
    \vspace{0.2cm}
    \item[\bluetri] by the good model assumption (Section 2.2), \\
    $\mathcal{L}_1(\theta,\tilde{\theta}) \approx tr\{-L''_{\tilde{\theta}}(\tilde{\theta}-\theta)(\tilde{\theta}-\theta)^T\}$
    \vspace{0.2cm}
    \item[\bluetri] $E_{\theta|y}(c_{\Theta}) \approx tr[-L''_{\tilde{\theta}}E_{\theta|y}\{(\theta-\tilde{\theta})(\theta-\tilde{\theta})^T\}] + E_{\theta|y}\{\mathcal{L}_2(y,\theta)\} + E_{\theta|y}\{D(\theta)-D(\tilde{\theta})\}$
\end{itemize}


\end{frame}



%%%%%%%%%%%%%%%%%%%%%%
\begin{frame}[t]
\frametitle{An approximate decision theory for DIC}

\begin{itemize}
    \item[\bluetri] Using the posterior mean $\bar{\theta}$ \\ $E_{\theta|y}(c_{\Theta}) \approx tr(-L''_{\bar{\theta}}V) + E_{\theta|y}\{\mathcal{L}_2(y,\theta)\} + p_D$ \\
    where $V$ is defined as the posterior covariance of $\theta$, and $p_D = \bar{D} - D(\bar{\theta})$  
    \vspace{0.2cm}
    \item[\bluetri] $\mathcal{L}_2(y,\theta) = E_{Y_{rep}|\theta}[-2\log\{p(Y_{rep}|\theta)\}] + 2\log\{p(y|\theta)\}$ \\ and so $E_Y[E_{\theta|Y}\{\mathcal{L}_2(Y,\theta)\}] = E_{\theta}[E_{Y|\theta}\{\mathcal{L}_2(Y,\theta)\}]=0$
    \vspace{0.2cm}
    \item[\bluetri] We have already shown that $p_D \approx tr(-L''_{\bar{\theta}}V)$ \\ the expected posterior loss is $D(\bar{\theta})+E_{\theta|y}(c_{\Theta}) \approx D(\bar{\theta}) + 2p_D = DIC$ \\
    neglecting a term $E_{\theta|y}\{\mathcal{L}_2(y,\theta)\}=0$
    \vspace{0.2cm}
    \item[\bluetri] The derivation has assumed that $D$ is an unstandardized deviance
\end{itemize}


\end{frame}


%%%%%%%%%%%%%%%%%%%%%%%
%%%%%%%%%%%%%%%%%%%%%%%
%%%%%%%%%%%%%%%%%%%%%%%
\section[Outro]{Outro}
\subsection*{Outro}



%%%%%%%%%%%%%%%%%%%%%%

\begin{frame}[t]
\frametitle{Finally...!}

\begin{itemize}
   \item[\bluetri] Code Link
    \item[] \hspace{1.23cm} (\href{https://github.com/HajinKang/Bayesian/blob/master/Bayesian_Seminar5.tex}{https://github.com/HajinKang/Bayesian/blob/master/Bayesian\_Seminar5.tex})
   \vspace{0.2cm}
   \item[\bluetri] Let's consider online \LaTeX editor (\href {https://www.overleaf.com}{https://www.overleaf.com})
\end{itemize}

\begin{figure}[h]
	\begin{center}
		\includegraphics[width=0.25\textwidth,keepaspectratio=true]{UploadFiles/overleaf.png}
	\end{center}
\end{figure}

\footnotetext[1]{source: \href{https://www.overleaf.com}{website of online \LaTeX editor overleaf}}

  \vspace{1cm}

\end{frame}
%%%%%%%%%%%%%%%%%%%%%%%
%%%%%%%%%%%%%%%%%%%%%%%
%%%%%%%%%%%%%%%%%%%%%%%
%\section[]{}
%%\subsection*{Appendix}
%%%%%%%%%%%%%%%%%%%%%%%%
%\begin{frame}
%\frametitle{}
%
%$$ \mbox{\Huge Thank you!} $$
%       
%\end{frame}





%%%%%%%%%%%%%%%%%%%%%

% The end of slides
\end{document} 